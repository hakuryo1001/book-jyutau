% IDC - ideographic description characters
% https://en.wikipedia.org/wiki/Chinese_character_description_languages#Ideographic_Description_Sequences

\newcommand{\superimpose}[2]{{%
  \ooalign{%
    \hfil$\m@th\text{#1}\@firstoftwo\text{#2}$\hfil\cr
    \hfil$\m@th\text{#1}\@secondoftwo\text{#2}$\hfil\cr
  }%
}}


% Define the \tb command

\newcommand{\tb}[2]{%
\scalebox{2}[1]{
\ooalign{%
    \hfil\raisebox{0.25em}{\text{\scalebox{0.33}{#1}}}\hfil\cr % Top text, squished and raised
    \hfil\raisebox{-0.25em}{\text{\scalebox{0.33}{#2}}}\hfil\cr % Bottom text, squished and lowered
  }%
  }
}

% The \lr command - can be combined with \tb
\newcommand{\lr}[2]{
  \scalebox{0.5}[1.0]{#1}\scalebox{0.5}[1.0]{#2}\!\!
}

% Define the \ul command for upper left positioning - for characters like 疒
\newcommand{\ul}[2]{%
  \ooalign{%
    \hfil#1\hfil\cr  % Top text (unscaled)
    \hfil\hspace{0.3em}\scalebox{0.8}{#2}\cr % Bottom text (scaled and raised)
    % \hfil\raisebox{0.2em}{\scalebox{0.5}{#2}}\hfil\cr % Bottom text (scaled and raised)
  }%
}

% Define the \tone command for upper right positioning of a diacritic
\newcommand{\tone}[2]{%
  \ooalign{%
    \hfil#1\hfil\cr  % Main text (unscaled)
    \hfil\hspace{0.9em}\raisebox{0.3em}{\scalebox{0.8}{#2}}\hfil\cr % Tone mark (scaled and raised)
  }%
}
